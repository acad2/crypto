% IACR Transactions CLASS DOCUMENTATION -- version 0.23 (27 May 2016)
% Written by Gaetan Leurent gaetan.leurent@inria.fr (2016)
%
% To the extent possible under law, the author(s) have dedicated all
% copyright and related and neighboring rights to this software to the
% public domain worldwide. This software is distributed without any
% warranty.
%
% You should have received a copy of the CC0 Public Domain Dedication
% along with this software. If not, see
% <http://creativecommons.org/publicdomain/zero/1.0/>.

\documentclass[preprint]{iacrtrans}
\usepackage[utf8]{inputenc}

\usepackage{graphicx}
\graphicspath{{images/}} % end dirs with `/'
\usepackage{tikz}
\usetikzlibrary{arrows}
\usetikzlibrary{arrows.meta}
\usetikzlibrary{positioning}
\usetikzlibrary{calc}
\usetikzlibrary{backgrounds}
\usetikzlibrary{arrows}
\usetikzlibrary{crypto.symbols}
\tikzset{shadows=no}        % Option: add shadows to XOR, ADD, etc.

\author{Ella Rose\inst{1} \and Biv\inst{2}}
\institute{Somewhere in California \email{python_pride@protonmail.com} \and Somewhere in France}
\title[A new block cipher: \texttt{SOME NAME}]{A new block cipher: \texttt{SOME NAME}}
\subtitle{Tentative of describing it.}

\begin{document}

\maketitle

% use optional argument because the \LaTeX command breaks the PDF keywords
\keywords[Block Cipher, Cryptanalysis, ARX]{Block Cipher \and Cryptanalysis \and Add-Rotate-Xor}

\begin{abstract}
  This document is a quick presentation of a self made block cipher.
\end{abstract}

\section*{Introduction}

% \subsection*{\textcolor{red!70!black}{FAQ:} Converting \texttt{llncs} papers to \texttt{iacrtrans}}

\section{Main Idea}

\begin{figure}[!ht]
	\centering
	\resizebox{0.8\textwidth}{!}{%
				\begin{tikzpicture}
			\tikzset{
				mynode/.style={rectangle,draw=black, thick, inner sep=1em, minimum size=3em, text centered, minimum width=2cm},
				mynodep/.style={rectangle,text centered, minimum width=0.5cm, minimum height=1.1cm},
				myarrow/.style={->, >=latex', shorten >=1pt, very thick},
				mylabel/.style={text width=7em, text centered}
			}


			\foreach \x in {0, ..., 63} {
				\draw [black] (\x,20) -- (\x,19);
			}

			\def \toside {4}
			\foreach \x in {0, ..., 15} {
				\draw [black,fill=white] (-0.25+\x*\toside,19.5) -- (3.25+\x*\toside,19.5) -- (3.25+\x*\toside,18.5) -- (-0.25+\x*\toside,18.5) -- cycle;
			}

			\foreach \x in {0, ..., 15} {
				\draw [black,fill=white] (-0.25+\x*\toside,10.5) -- (3.25+\x*\toside,10.5) -- (3.25+\x*\toside,9.5) -- (-0.25+\x*\toside,9.5) -- cycle;
			}

			\foreach \i [count=\xi] in {7, 12, 14, 9, 2, 1, 5, 15, 11, 6, 13, 0, 4, 8, 10, 3} {
				\draw (\xi*\toside-\toside+1.5,18.5) -- (\xi*\toside-\toside+1.5,18) -- (\i*\toside+1.5,11) -- (\i*\toside+1.5,10.5);
			}

			% \node[mynodep] (t1) {$t_1$};
			% \node[mynodep,right=0.5cm of t1] (Op1) {$\bigoplus$};
			% \node[mynodep,right=0.5cm of Op1] (t1p) {$t_1'$};
			% \node[mynodep,right=of t1p] (A1) {$= \Delta_1$};

			% \node[mynode, below=0.5cm of t1] (f) {$f$};
			% \node[mynode, below=0.5cm of t1p] (fp) {$f$};

			% \node[mynodep,below=0.5cm of f] (t2) {$t_2$};
			% \node[mynodep,right=0.5cm of t2] (Op2) {$\bigoplus$};
			% \node[mynodep,right=0.5cm of Op2] (t2p) {$t_2'$};
			% \node[mynodep,right=of t2p] (A2) {$= \Delta_2$};

			% \draw[myarrow] (t1.south) -- (f.north);
			% \draw[myarrow] (t1p.south) -- (fp.north);
			% \draw[myarrow] (f.south) -- (t2.north);
			% \draw[myarrow] (fp.south) -- (t2p.north);
		\end{tikzpicture}

	}%
	\caption{Block cipher construction}
	\label{fig:orbitals}
\end{figure}


\section{Security Analysis}



\section*{Thanks}

We would like to thank people who helped design and improve this block cipher:
MickLH.

\section*{Changes}

\begin{description}
\item[v 0.0000001] First public version
\end{description}


\end{document}
